\documentclass[a4paper,11pt,showtrims]{memoir}
\begin{document}

An Investigation of Australian Rainfall using Extreme Value Theory\\

Extreme rainfall and associated flooding cause major socio-economic disruptions to communities. Some of the worst impacts include loss of life and infrastructure damage. To help mitigate these potential impacts, we require an understanding of the risk of these events. In particular, we need to be able to estimate the probability that an extreme rainfall event will impact nearby locations within a region, not just a single point location. With this in mind, we use methods from extreme value theory to fit statistical models for Australian daily rainfall extremes. We build upon the existing literature by helping to create a bridge between technical statistical ideas and applications in a climate setting. In the first part of this talk, we provide an overview of the applications considered in this thesis. \\

% These applications include developing an understanding of how the large scale climate driver, El Niño Southern Oscillation (ENSO), influences the distribution of daily rainfall extremes. \\

In the latter part of the talk, we present a regionalisation of Australia based on the dependence of rainfall extremes. The regions are formed using hierarchical clustering and a distance measure based on bivariate extremal dependence, that is estimated using the F-madogram. Using the F-madogram is powerful, as no information about climate or topography is needed to form spatially homogeneous clusters. Additionally, we are free from distributional assumptions as the F-madogram distances can be estimated non-parametrically from raw maxima. Developing this regionalisation has improved our understanding of local dependence of rainfall extremes and helped to ensure the validity of common assumptions used when modelling spatial extremes.\\



% In addition to presenting the regionalisation, we highlighted key methodological considerations when using the F-madogram dissimilarity for unsupervised learning.



% where regions are formed based on the dependence of rainfall extremes.  This regionalisation was created using only the annual maximum rainfall observations at stations




% where regions are formed according to the dependence of rainfall extremes. In Australia, a single, fixed dependence structure for spatial models of rainfall extremes is unrealistic. This is due to the country size, variations in climate and complexity of topography. However, it is not uncommon within spatial extremes to assume a fixed dependence structure across a domain, which is clearly unsuitable. To account for this, we present a regionalisation of Australia, where regions are identified according to similar dependence of rainfall extremes. Using these regions, we show there that are many different dependence structures for extreme rainfall in Australia. This shows that caution is needed when grouping rainfall stations to fit a max-stable process, as the dependence structure may be variable. \\

% We discuss whether this modelling approach is suitable for an Australian wide scale. The dependence structure of max-stable processes can be inflexible for regions of large geographical size. To utilise the methods presented in the first part of the talk, we present a clustering and classification method that identifies regions of similar extremal dependence. Using these regions, we show there that are many different dependence structures for extreme rainfall in Australia. This shows that caution is needed when grouping rainfall stations to fit a max-stable process, as the dependence structure may be variable.

% ------------------------------------------------------

% In particular, we require an understanding of how extreme rainfall will impact nearby locations within a region, not just a single point location. \\


% In the first part of this talk, we give an overview of



% In the first part of this talk, we consider a flash flood that occurred in the city of Toowoomba, Australia in 2011. We assess the probability of this event occurring by fitting and simulating from a max-stable process. Max-stable processes are a natural extension of univariate extreme value theory for modelling extremes in continuous space with dependence. We show that the probability of this flash flood was more likely given the state of the climate.\\

% In the second part of this talk, we discuss whether this modelling approach is suitable for an Australian wide scale. The dependence structure of max-stable processes can be inflexible for regions of large geographical size. To utilise the methods presented in the first part of the talk, we present a clustering and classification method that identifies regions of similar extremal dependence. Using these regions, we show there that are many different dependence structures for extreme rainfall in Australia. This shows that caution is needed when grouping rainfall stations to fit a max-stable process, as the dependence structure may be variable.

% ---------------------------------------

% In this thesis, we use extreme value theory to fit statistical models for Australian daily rainfall extremes. We build upon the existing literature by challenging the basic assumptions of these models when applied in a climate setting. The types of applications we consider range from univariate extreme value theory, to spatial extremes with dependence, and finally to an investigation of extremal dependence. The combined application content provides an in-depth investigation into our understanding of Australian rainfall extremes and the risk posed by these extreme events.\\

% %In the first content chapter of this thesis we challenge the assumption of stationarity. 
% We consider how large scale climate drivers, like El Ni\~{n}o Southern Oscillation (ENSO), can influence the distribution of rainfall extremes. Using observations of daily rainfall from station data, we quantify the magnitude and spatial influence of ENSO on the distribution of seasonal maximum daily rainfall. We contrast these results obtained from an at-station analysis, to those from a simple spatial model, ultimately producing maps of the region of ENSO influence.\\ 

% %In the second content chapter, we challenge using an at station approach to assess risk, compared with a spatial approach. 
% We then transition to an application where we use max-stable processes to model rainfall extremes in continuous space with dependence. We fit a max-stable process to the annual maximum daily rainfall in South East Queensland and simulate the extreme precipitation field. %We have chosen to use max-stable processes as they are the natural extension of univariate extreme value theory to extremes in continuous space with dependence. 
% We quantify the severity of a historical flash flood in this region, showing that the probability of this event was significantly higher given the phase of ENSO.% by considering the extreme preciptation field. 
% \\

% % By using a model for spatial extremes with dependence our ability to detect non-stationarity improves compared to a model fit to a single station. We also make better use of the available data. This has additional benefits of reducing model uncertainty.\\

% % In the third content chapter, 
% Finally, we consider variation in the dependence behaviour of daily rainfall extremes. For Australia, a single dependence structure for spatial models of rainfall extremes is unrealistic. This is due to the country size, variations in climate and complexity of topography. To account for this, we present a regionalisation of Australia, where regions are identified according to similar dependence of rainfall extremes.\\

% The overarching goal of this thesis is to improve our understanding of the risk posed by extreme rainfall events in Australia. We achieve this by utilising spatial, statistical models. However, we acknowledge that using extreme value theory for modelling real world applications can have challenges in practice. We highlight these practical considerations in our applications, so that other researchers are aware of the advantages of this kind of modelling, but also the practical limitations.

\end{document}